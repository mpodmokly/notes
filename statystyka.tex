\documentclass[11pt, leqno]{scrartcl}
\usepackage{polski}
\usepackage[polish]{babel}

\usepackage{graphicx, float, caption, subcaption}
\usepackage{tabularx, multirow, hyperref, enumitem}
\usepackage{listings, xcolor}
\usepackage{amsmath, amssymb}
\usepackage{amsthm}

\newtheoremstyle{mydefinition}
    {1.2em}{1.2em}{}{}{\bfseries}{}{0pt}
    {\thmname{#1} \thmnumber{#2.}\thmnote{ #3.}\\[0.5em]}
\theoremstyle{mydefinition}
\newtheorem{definition}{Definicja}[subsection]

\newtheoremstyle{mytheorem}
    {1.2em}{1.2em}{}{}{\bfseries}{}{0pt}
    {\thmname{#1} \thmnumber{#2.}\thmnote{ #3.}\\[0.5em]}
\theoremstyle{mytheorem}
\newtheorem{theorem}{Twierdzenie}[subsection]

% \newtheoremstyle{style name}
% {space above}
% {space below}
% {body font}
% {indent amount}
% {head font}
% {head punct}
% {after head space}
% {head spec}

\title{Rachunek prawdopodobieństwa i statystyka}
\author{Mateusz Podmokły III rok Informatyka WI}
\date{semestr zimowy 2025}

\begin{document}
    \maketitle
    \section{Podstawy rachunku prawdopodobieństwa}
    \subsection{Przestrzeń probabilistyczna}
    \begin{definition}[Przestrzeń zdarzeń elementarnych]
        Niepusty zbiór $\Omega$ wszystkich możliwych wyników
        doświadczenia losowego. Jego elementy to zdarzenia elementarne.
    \end{definition}
    \begin{definition}[$\sigma$-algebra zdarzeń]
        Podrodzina $\Sigma$ w rodzinie wszystkich podzbiorów $\Omega$
        o następujących właściwościach:
        \begin{enumerate}
            \item $\Omega \in \Sigma$
            \item Jeśli $A \in \Sigma$ to $A'=\Omega \setminus A \in
                \Sigma$
            \item Dla dowolnego ciągu zbiorów $A_1,A_2,\dots$ takiego,
                że $A_i \in \Sigma$ dla $i \in \mathbb{N}$, zachodzi
                \[
                    \bigcup_{i=1}^{\infty}A_i \in \Sigma
                \]
        \end{enumerate}
        Jej elementy to zdarzenia losowe.
    \end{definition}
    \noindent
    Własności zdarzeń:
    \begin{enumerate}
        \item $\emptyset \in \Sigma$
        \item Jeśli $A_1,A_2,\dots,A_n \in \Sigma$, to
            $A_1 \cup A_2 \cup \dots \cup A_n \in \Sigma$
        \item Dla dowolnego ciągu $A_1,A_2,\dots$ takiego, że
            $A_i \in \Sigma$ dla $i \in \mathbb{N}$, zachodzi
            \[
                \bigcap_{i=1}^{\infty}A_i \in \Sigma
            \]
        \item Jeśli $A,B \in \Sigma$ to $A \setminus B \in \Sigma$
    \end{enumerate}
    Określmy niezbędną terminologię:
    \begin{description}
        \item[$\emptyset$] -- zdarzenie niemożliwe
        \item[$\Omega$] -- zdarzenie pewne
        \item[$A'=\Omega \setminus A$] -- dopełnienie zdarzenia $A$
        \item[$A \cap B= \emptyset$] -- zdarznia wzajemnie się
            wykluczają.
    \end{description}
    \begin{definition}[Miara probabilistyczna (rozkład
        prawdopodobieństwa)]
        W przestrzeni $\Omega$ z $\sigma$-algebrą zdarzeń $\Sigma$
        dowolne odwzorowanie
        \[
            P:\Sigma \to [0,1]
        \]
        spełniające warunki:
        \begin{enumerate}
            \item $P(\Omega)=1$
            \item Dla dowolnego ciągu zdarzeń $A_1,A_2,\dots$ takiego,
                że $A_i \in \Sigma$ dla $i \in \mathbb{N}$ oraz
                $A_i \cap A_j=\emptyset$ dla $i \neq j$ zachodzi
                \[
                    P\left(\bigcup_{i=1}^{\infty}A_i\right)=
                    \sum_{i=1}^{\infty}P(A_i)
                \]
        \end{enumerate}
    \end{definition}
    Własności prawdopodobieństwa:
    \begin{enumerate}
        \item $P(\emptyset)$=0
        \item Jeśli skończony ciąg zdarzeń $A_1,A_2,\dots,A_n$ spełnia
            warunek
            \[
                A_i \cap A_j = \emptyset \text{ dla }i \neq j
            \]
            to
            \[
                P(A_1 \cup A_2 \cup \dots \cup A_n)=
                P(A_1)+P(A_2)+\dots +P(A_n)
            \]
        \item Dla dowolnego zdarzenia $A$ zachodzi
            \[
                P(A')=1-P(A)
            \]
        \item Dla dowolnych zdarzeń $A$ i $B$ zachodzi
            \[
                P(A \cup B)=P(A)+P(B)-P(A \cap B)
            \]
        \item Jeśli $A \subset B$ to $P(A)\leq P(B)$
        \item Jeśli zdarzenia $A_1,A_2,\dots$ tworzą ciąg
            wstępujący, tzn.
            \[
                A_1 \subset A_2 \subset \dots
            \]
            to
            \[
                P\left( \bigcup_{i=1}^{\infty}A_i \right)=
                \lim_{i \to \infty}P(A_i)
            \]
        \item Jeśli zdarzenia $A_1,A_2,\dots$ tworzą ciąg
            zstępujący, tzn.
            \[
                A_1 \supset A_2 \supset \dots
            \]
            to
            \[
                P\left( \bigcap_{i=1}^{\infty}A_i \right)=
                \lim_{i \to \infty}P(A_i)
            \]
    \end{enumerate}
    \begin{definition}[Przestrzeń probabilistyczna]
        Trójka $(\Omega,\Sigma,P)$, gdzie:
        \begin{description}
            \item[$\Omega$] -- niepusty zbiór,
            \item[$\Sigma$] -- $\sigma$-algebra w $\Omega$,
            \item[$P$] -- miara probabilistyczna.
        \end{description}
    \end{definition}
    Liczbę $P(A)$ nazywamy prawdopodobieństwem zdarzenia $A$.

    \subsection{Prawdopodobieństwo warunkowe}
    \begin{definition}[Prawdopodobieństwo warunkowe]
        Liczba określona wzorem
        \[
            P(A \mid B)=\frac{P(A \cap B)}{P(B)}
        \]
        gdzie
        \begin{description}
            \item[$A,B \subset \Omega$] -- zdarzenia,
            \item[$P(B)>0$.]
        \end{description}
        Jest to prawdopodobieństwo $A$ pod warunkiem $B$.
    \end{definition}
    \begin{definition}[Układ zupełny zdarzeń]
        Skończony lub nieskończony ciąg zdarzeń $A_1,A_2,\dots$ jeśli
        zdarzenia w ciągu parami wzajemnie się wykluczają, tzn.
        \[
            A_i \cap A_j =\emptyset, \quad i \neq j
        \]
        oraz zachodzi
        \[
            \Omega=\bigcup_{i}A_i
        \]
    \end{definition}
    \begin{theorem}[Twierdzenie o prawdopodobieństwie całkowitym]
        Jeśli zdarzenia $A_1,A_2,\dots$ tworzą układ zupełny oraz
        $P(A_i)>0$ dla $i \in \mathbb{N}$, to dla dowolnego
        zdarzenia $B$ zachodzi
        \[
            P(B)=\sum_{i}P(B \mid A_i)P(A_i)
        \]
    \end{theorem}
    \begin{theorem}[Twierdzenie Bayesa]
        Jeśli zdarzenia $A_i$ tworzą układ zupełny taki, że
        $P(A_i)>0$ dla $i \in \mathbb{N}$, a $B$ jest zdarzeniem
        takim, że $P(B)>0$, to dla dowolnego $k$ zachodzi
        \[
            P(A_k \mid B)=\frac{P(B \mid A_k)P(A_k)}
            {\sum_{i}P(B \mid A_i)P(A_i)}
        \]
    \end{theorem}
    
    \subsection{Niezależność zdarzeń}
    \begin{definition}[Niezależność zdarzeń]
        Zdarzenia $A$ i $B$ nazywamy niezależnymi jeśli
        \[
            P(A,B)=P(A) \cdot P(B)
        \]
    \end{definition}
\end{document}
