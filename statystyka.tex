\documentclass[11pt, leqno]{scrartcl}
\usepackage{polski}
\usepackage[polish]{babel}

\usepackage{graphicx, float, caption, subcaption}
\usepackage{tabularx, multirow, hyperref, enumitem}
\usepackage{listings, xcolor}
\usepackage{amsmath, amssymb}
\usepackage{amsthm}

\theoremstyle{definition}
\newtheorem{definition}{Definicja}[section]

\title{Rachunek prawdopodobieństwa i statystyka}
\author{Mateusz Podmokły III rok Informatyka WI}
\date{semestr zimowy 2025}

\begin{document}
    \maketitle
    \section{Podstawy rachunku prawdopodobieństwa}
    \begin{definition}
        \textbf{Przestrzeń zdarzeń elementarnych} to niepusty zbiór
        $\Omega$ wszystkich możliwych wyników doświadczenia losowego.
    \end{definition}
    \noindent
    Jego elementy to zdarzenia elementarne.
    \begin{definition}
        \textbf{$\sigma$-algebra zdarzeń} to podrodzina $\Sigma$
        w rodzinie wszystkich podzbiorów $\Omega$ o następujących
        właściwościach:
        \begin{enumerate}
            \item $\Omega \in \Sigma$
            \item Jeśli $A \in \Sigma$ to $A'=\Omega \setminus A \in
                \Sigma$
            \item Dla dowolnego ciągu zbiorów $A_1,A_2,\dots$ takiego,
                że $A_i \in \Sigma$ dla $i \in \mathbb{N}$, zachodzi
                \[
                    \bigcup_{i=1}^{\infty}A_i \in \Sigma
                \]
        \end{enumerate}
        Jej elementy to zdarzenia losowe.
    \end{definition}
    \noindent
    Własności zdarzeń:
    \begin{enumerate}
        \item $\emptyset \in \Sigma$
        \item Jeśli $A_1,A_2,\dots,A_n \in \Sigma$, to
            $A_1 \cup A_2 \cup \dots \cup A_n \in \Sigma$
        \item Dla dowolnego ciągu $A_1,A_2,\dots$ takiego, że
            $A_i \in \Sigma$ dla $i \in \mathbb{N}$, zachodzi
            \[
                \bigcap_{i=1}^{\infty}A_i \in \Sigma
            \]
        \item Jeśli $A,B \in \Sigma$ to $A \setminus B \in \Sigma$
    \end{enumerate}
    Określmy niezbędną terminologię:
    \begin{description}
        \item[$\emptyset$] -- zdarzenie niemożliwe
        \item[$\Omega$] -- zdarzenie pewne
        \item[$A'=\Omega \setminus A$] -- dopełnienie zdarzenia $A$
        \item[$A \cap B= \emptyset$] -- zdarznia wzajemnie się
            wykluczają.
    \end{description}
    \begin{definition}
        \textbf{Miara probabilistyczna} (rozkład prawdopodobieństwa)
        to w przestrzeni $\Omega$ z $\sigma$-algebrą zdarzeń $\Sigma$
        dowolne odwzorowanie
        \[
            P:\Sigma \to [0,1]
        \]
        spełniające warunki:
        \begin{enumerate}
            \item $P(\Omega)=1$
            \item Dla dowolnego ciągu zdarzeń $A_1,A_2,\dots$ takiego,
                że $A_i \in \Sigma$ dla $i \in \mathbb{N}$ oraz
                $A_i \cap A_j=\emptyset$ dla $i \neq j$ zachodzi
                \[
                    P\left(\bigcup_{i=1}^{\infty}A_i\right)=
                    \sum_{i=1}^{\infty}P(A_i)
                \]
        \end{enumerate}
    \end{definition}
    Własności prawdopodobieństwa:
    \begin{enumerate}
        \item $P(\emptyset)$=0
        \item Jeśli skończony ciąg zdarzeń $A_1,A_2,\dots,A_n$ spełnia
            warunek
            \[
                A_i \cap A_j = \emptyset \text{ dla }i \neq j
            \]
            to
            \[
                P(A_1 \cup A_2 \cup \dots \cup A_n)=
                P(A_1)+P(A_2)+\dots +P(A_n)
            \]
        \item Dla dowolnego zdarzenia $A$ zachodzi
            \[
                P(A')=1-P(A)
            \]
        \item Dla dowolnych zdarzeń $A$ i $B$ zachodzi
            \[
                P(A \cup B)=P(A)+P(B)-P(A \cap B)
            \]
        \item Jeśli $A \subset B$ to $P(A)\leq P(B)$
        \item Jeśli zdarzenia $A_1,A_2,\dots$ tworzą ciąg
            wstępujący, tzn.
            \[
                A_1 \subset A_2 \subset \dots
            \]
            to
            \[
                P\left( \bigcup_{i=1}^{\infty}A_i \right)=
                \lim_{i \to \infty}P(A_i)
            \]
        \item Jeśli zdarzenia $A_1,A_2,\dots$ tworzą ciąg
            zstępujący, tzn.
            \[
                A_1 \supset A_2 \supset \dots
            \]
            to
            \[
                P\left( \bigcap_{i=1}^{\infty}A_i \right)=
                \lim_{i \to \infty}P(A_i)
            \]
    \end{enumerate}
    \begin{definition}
        \textbf{Przestrzeń probabilistyczna} to trójka
        $(\Omega,\Sigma,P)$, gdzie:
        \begin{description}
            \item[$\Omega$] -- niepusty zbiór,
            \item[$\Sigma$] -- $\sigma$-algebra w $\Omega$,
            \item[$P$] -- miara probabilistyczna.
        \end{description}
    \end{definition}
    Liczbę $P(A)$ nazywamy prawdopodobieństwem zdarzenia $A$.

    \section{Prawdopodobieństwo warunkowe}
    \begin{definition}
        \textbf{Prawdopodobieństwo warunkowe} to liczba określona
        wzorem
        \[
            P(A \mid B)=\frac{P(A \cap B)}{P(B)}
        \]
        gdzie
        \begin{description}
            \item[$A,B \subset \Omega$] -- zdarzenia,
            \item[$P(B)>0$].
        \end{description}
        Jest to prawdopodobieństwo $A$ pod warunkiem $B$.
    \end{definition}

    \section{Literatura}
\end{document}
