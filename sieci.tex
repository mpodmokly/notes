\documentclass[11pt, leqno]{scrartcl}
\usepackage{polski}
\usepackage[polish]{babel}

\usepackage{graphicx, float, caption, subcaption}
\usepackage{tabularx, multirow, hyperref, enumitem}
\usepackage{listings, xcolor}
\usepackage{amsmath, amssymb}
\usepackage{amsthm}

\hypersetup{
    colorlinks=true,
    linkcolor=black,
    urlcolor=black,
    citecolor=black
}

\makeatletter
\renewcommand{\thebibliography}[1]{%
  \list{\@biblabel{\@arabic\c@enumiv}}%
       {\settowidth\labelwidth{\@biblabel{#1}}%
        \leftmargin\labelwidth
        \advance\leftmargin\labelsep
        \usecounter{enumiv}%
        \let\p@enumiv\@empty
        \renewcommand\theenumiv{\@arabic\c@enumiv}}%
  \sloppy\clubpenalty4000\widowpenalty4000%
  \sfcode`\.=1000\relax}
\makeatother

\title{Sieci komputerowe}
\author{Mateusz Podmokły III rok Informatyka WI}
\date{semestr zimowy 2025}

\begin{document}
    \maketitle
    \tableofcontents
    \newpage

    \section{Spanning Tree Protocol (STP)}
    \subsection{Wybór Root Bridge}
    \begin{center}
        Bridge ID (BID) = priorytet + MAC
    \end{center}
    Najpierw używany jest priorytet switcha, ewentualnie, jak remis,
    to MAC. Root Bridge zostaje urządzenie o najmniejszej wartości
    BID + MAC, czyli najwyższy priorytet.
    
    \subsection{Wybór Root Port}
    Dla każdego switcha wybierany jest jeden Root Port, który prowadzi
    najtańszą ścieżką do Root Bridge. Do rozstrzygania remisów używane
    są następujące kryteria:
    \begin{enumerate}
        \item Root Path Cost - najmniejszy łączny koszt ścieżki do
            Root Bridge
        \item Lowest Sender BID - priorytet nadrzędnego switcha
        \item Lowest Port ID (PID) - numer portu nadawcy
        \item Local PID - lokalny numer portu
    \end{enumerate}

    \subsection{Wybór Designated Port}
    Wszystkie porty Root Bridge są Designated Port. Dla każdego
    segmentu sieci wybierany jest jeden Designated Port na podstawie
    następujących kryteriów:
    \begin{enumerate}
        \item Root Path Cost - najmniejszy łączny koszt ścieżki do
            Root Bridge
        \item Lowest BID - priorytet switcha
        \item Lowest PID - numer portu
    \end{enumerate}

    \subsection{Wybór Blocking Port}
    Wszystkie pozostałe porty zostają Blocking Port.

    \section{Tablica forwardingu (FDB)}
    \subsection{Uczenie się (Learning)}
    Switch odbiera ramkę Ethernet od danego urządzenia z jakiegoś swojego portu.
    Jeżeli danego adresu MAC nie ma jeszcze w tablicy, to switch dodaje wpis do
    tablicy forwardingu w postaci MAC $\rightarrow$ port.

    \subsection{Przekazywanie (Forwarding)}
    Jeżeli switch ma adres docelowy w tablicy forwardingu, to wie na który port
    przesłać ramkę. Natomiast jeżeli nie ma, to przesyła ją na wszystkie porty
    z wyjątkiem portu, z którego przyszła.

    \section{Address Resolution Protocol (ARP)}
    \subsection{Sprawdzenie sieci}
    Urządzenie sprawdza, czy docelowy adres IP jest w tej samej sieci. Jeśli tak, to
    wysyła zapytanie ARP o adres docelowy, a jeśli nie to wysyła zapytanie ARP
    o adres bramy domyślnej i później tam prześle pakiet. Dalej ARP działa też
    w kolejnych sieciach na drodze pakietu zgodnie z tablicami routingu.

    \subsection{Żądanie ARP}
    Urządzenie wysyła komunikat rozgłoszeniowy \texttt{ARP Request} (broadcast) do
    wszystkich w sieci lokalnej pytając kto ma docelowy adres IP, którego poszukuje.
    Oczekuje adresu MAC tego urządzenia.

    \subsection{Odpowiedź ARP}
    Tylko urządzenie o wskazanym adresie IP odpowiada komunikatem bezpośrednim
    \texttt{ARP Reply} wysyłając swój adres MAC. Przy okazji zapisuje w swojej
    tablicy ARP adres MAC urządzenia, które przysłało zapytanie. Pozostałe urządzenia
    w sieci (świadkowie) ignorują zapytanie.

    \subsection{Zapis w tablicy ARP}
    Urządzenie źródłowe zapisuje otrzymaną parę IP $\rightarrow$ MAC w swojej tablicy
    ARP. Przy następnej komunikacji już nie musi ponownie wysyłać zapytania.

    \subsection{Proxy ARP}
    Jeżeli urządzenie myśli, że adres docelowy jest w tej samej sieci (np. przez
    maskę), to wysyła zapytanie ARP w sieci lokalnej. Wtedy, w przypadku włączonego
    Proxy ARP, router odpowiada na to zapytanie, że to jego adres IP, jeżeli wie, że
    docelowy adres IP jest osiągalny przez jego inny interfejs.

    \section{Routing}
    \subsection{EIGRP}
    Protokół dla każdej ścieżki do danej sieci oblicza wartość metryki i wybiera
    ścieżkę z najniższą metryką, wg wzoru
    \[
        M=256 \cdot (K_1B+K_3D)
    \]
    gdzie
    \[
        B=\frac{10^7}{min(\text{bandwith na ścieżce})(\text{kbps})}
    \]
    \[
        D=\frac{\sum \text{delay}(\mu s)}{10}
    \]
    Jeżeli dwie ścieżki mają tę samą wartość metryki, to zazwyczaj zostaną ustawione
    obydwie jako opcja load balancing (chyba że w konfiguracji ustawiono inaczej).

    \subsection{OSPF}
    Wykorzystuje prostszą metrykę do wyznaczania najlepszej trasy, wg wzoru
    \[
        M=\sum\frac{10^8}{B}
    \]
    gdzie $B$ to przepustowość interfejsu. Sumujemy po wszystkich interfejsach
    wychodzących na trasie do sieci docelowej. Wybiera trasę z najniższą wartością
    metryki.

    \subsection{RIP}
    Jako metrykę wykorzystuje liczbę routerów (hopów) po drodze do sieci docelowej.
    Im mniej tym lepiej.
    
    \section{Fragmentacja IP}
    W momencie, kiedy wielkość pakietu IP przekracza Maximum Transmission Unit (MTU)
    łącza, pakiet jest dzielony na mniejsze takie, żeby wielkość danych razem
    z nagłówkiem IP (20B) zmieściła się na danym łączu. Jeżeli fragment nie mieści się
    w dalszej części trasy na jakimś łączu do jest dalej odpowiednio dzielony. Pakiet
    składany jest w całość dopiero przez hosta docelowego. Tabela przedstawia pola
    związane z fragmentacją pakietów:
    \begin{table}[H]
        \centering
        \renewcommand{\arraystretch}{1.7}
        \begin{tabular}{|c|c|c|c|c|}
            \hline
            \textbf{ID} & \textbf{R} & \textbf{DF} & \textbf{MF} & \textbf{Offset} \\
            \hline
            * & 0 & 0 & 1 & 0 \\
            \hline
            * & 0 & 0 & 0 & 122 \\
            \hline
        \end{tabular}
        \caption{Przykładowe wartości pól.}
    \end{table}
    \begin{description}
        \item[ID]- identyfikator pakietu,
        \item[R]- bit zarezerwowany,
        \item[DF]- \textit{Don't Fragment}, zakaz podziału pakietu,
        \item[MF]- \textit{More Fragments}, jeśli 1, to będzie więcej fragmentów
            pakietu, a 0 oznacza ostatni fragment,
        \item[Offset]- pozycja danego fragmentu w oryginalnym pakiecie, liczona
            w blokach po 8B.
    \end{description}

    \section{Literatura}
    \begin{thebibliography}{9}
        \bibitem{Zieliński} Zieliński, K. (2026). Sieci komputerowe. \textit{Wykłady
        prowadzone na Akademii Górniczo-Hutniczej w Krakowie}.
    \end{thebibliography}
\end{document}
