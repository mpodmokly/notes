\documentclass[11pt, leqno]{scrartcl}
\usepackage{polski}
\usepackage[polish]{babel}

\usepackage{graphicx, float, caption, subcaption}
\usepackage{tabularx, multirow, hyperref, enumitem}
\usepackage{listings, xcolor}
\usepackage{amsmath, amssymb}
\usepackage{amsthm}

\hypersetup{
    colorlinks=true,
    linkcolor=black,
    urlcolor=black,
    citecolor=black
}

\makeatletter
\renewcommand{\thebibliography}[1]{%
  \list{\@biblabel{\@arabic\c@enumiv}}%
       {\settowidth\labelwidth{\@biblabel{#1}}%
        \leftmargin\labelwidth
        \advance\leftmargin\labelsep
        \usecounter{enumiv}%
        \let\p@enumiv\@empty
        \renewcommand\theenumiv{\@arabic\c@enumiv}}%
  \sloppy\clubpenalty4000\widowpenalty4000%
  \sfcode`\.=1000\relax}
\makeatother

\title{Sieci komputerowe}
\author{Mateusz Podmokły III rok Informatyka WI}
\date{semestr zimowy 2025}

\begin{document}
    \maketitle
    \tableofcontents
    \newpage

    \section{Spanning Tree Protocol (STP)}
    \subsection{Wybór Root Bridge}
    priorytet = Bridge ID (BID) + MAC \\
    Najpierw używane jest BID, ewentualnie, jak remis, to MAC. Root
    Bridge zostaje urządzenie o najmniejszej wartości BID + MAC,
    czyli najwyższy priorytet.
    
    \subsection{Wybór Root Port}
    Dla każdego switcha wybierany jest jeden Root Port, który
    prowadzi najtańszą ścieżką do Root Bridge. Jeśli kilka portów
    ma ten sam koszt, to używane są następujące kryteria:
    \begin{enumerate}
        \item Root Path Cost - najmniejszy łączny koszt ścieżki do
            Root Bridge
        \item Lowest Sender BID - priorytet nadrzędnego switcha
        \item Lowest Port ID (PID) - numer portu nadawcy
        \item Local PID - numer lokalnego portu
    \end{enumerate}

    \subsection{Wybór Designated Port}
    Wszystkie porty Root Bridge są Designated Port. Dla każdego
    segmentu sieci wybierany jest jeden Designated Port na podstawie
    następujących kryteriów:
    \begin{enumerate}
        \item Root Path Cost - najmniejszy łączny koszt ścieżki do
            Root Bridge
        \item Lowest BID - priorytet switcha
        \item Lowest PID - numer portu
    \end{enumerate}

    \subsection{Wybór Blocking Port}
    Wszystkie pozostałe porty zostają Blocking Port.
\end{document}
