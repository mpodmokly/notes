\documentclass[11pt, leqno]{scrartcl}
\usepackage{polski}
\usepackage[polish]{babel}

\usepackage{graphicx, float, caption, subcaption}
\usepackage{tabularx, multirow, hyperref, enumitem}
\usepackage{listings, xcolor}
\usepackage{amsmath, amssymb}
\usepackage{amsthm}

\hypersetup{
    colorlinks=true,
    linkcolor=black,
    urlcolor=black,
    citecolor=black
}

\makeatletter
\renewcommand{\thebibliography}[1]{%
  \list{\@biblabel{\@arabic\c@enumiv}}%
       {\settowidth\labelwidth{\@biblabel{#1}}%
        \leftmargin\labelwidth
        \advance\leftmargin\labelsep
        \usecounter{enumiv}%
        \let\p@enumiv\@empty
        \renewcommand\theenumiv{\@arabic\c@enumiv}}%
  \sloppy\clubpenalty4000\widowpenalty4000%
  \sfcode`\.=1000\relax}
\makeatother

\title{Sieci komputerowe}
\author{Mateusz Podmokły III rok Informatyka WI}
\date{semestr zimowy 2025}

\begin{document}
    \maketitle
    \tableofcontents
    \newpage

    \section{Spanning Tree Protocol (STP)}
    \subsection{Wybór Root Bridge}
    \begin{center}
        Bridge ID (BID) = priorytet + MAC
    \end{center}
    Najpierw używany jest priorytet switcha, ewentualnie, jak remis,
    to MAC. Root Bridge zostaje urządzenie o najmniejszej wartości
    BID + MAC, czyli najwyższy priorytet.
    
    \subsection{Wybór Root Port}
    Dla każdego switcha wybierany jest jeden Root Port, który prowadzi
    najtańszą ścieżką do Root Bridge. Do rozstrzygania remisów używane
    są następujące kryteria:
    \begin{enumerate}
        \item Root Path Cost - najmniejszy łączny koszt ścieżki do
            Root Bridge
        \item Lowest Sender BID - priorytet nadrzędnego switcha
        \item Lowest Port ID (PID) - numer portu nadawcy
        \item Local PID - lokalny numer portu
    \end{enumerate}

    \subsection{Wybór Designated Port}
    Wszystkie porty Root Bridge są Designated Port. Dla każdego
    segmentu sieci wybierany jest jeden Designated Port na podstawie
    następujących kryteriów:
    \begin{enumerate}
        \item Root Path Cost - najmniejszy łączny koszt ścieżki do
            Root Bridge
        \item Lowest BID - priorytet switcha
        \item Lowest PID - numer portu
    \end{enumerate}

    \subsection{Wybór Blocking Port}
    Wszystkie pozostałe porty zostają Blocking Port.

    \section{Rodzaje adresów fizycznych}
    \subsection{Unicast}
    \subsection{Multicast}
    \subsection{Broadcast}
    FF:FF:FF:FF:FF:FF
    \section{Sposoby przełączania ramek w switchach}
    \subsection{Store-and-forward}
    \subsection{Cut-through}
    \subsubsection{Fast-forward}
    \subsubsection{Fragment-free}

    \section{Kodowania}
    \subsection{NRZ (Non return to zero)}
    \subsection{NRZ-I (Non return to zero inverted)}
    \subsection{Manchester}
    Kodowanie zboczem 0 zbocze malejące, 1 zbocze rosnące.
    \subsection{Manchester różnicowy}
    Też zbocze, ale 0 powtarza poprzednie zbocze, a 1 odwraca
    poprzednie zbocze.

    \section{Trunkowanie VLANów}
    \subsection{TDM (Time division multiplexing)}
    \subsection{Filtrowanie}
    Tablica MAC-VLAN
    \subsection{Tagowanie}
    VLAN TAG do ramki Ehternet między adres nadawcy, a typ. Ma
    rozmiar 4B.

    \section{CSMA/CD}
    Carrier Sense Multiple Access with Collision Detection.
    \subsection{Carrier Sense (CS)}
    Każda stacja cały czas monitoruje medium.
    \subsection{Multiple Access (MA)}
    Każda stacja, która widzi wolne medium, może nadawać. Czeka
    InterFrame Gap (IFG) równe czasowi przesłania 96 bitów. Występują
    kolizje, jeżeli dwie stacje nadają naraz.
    \subsection{Collision Detection (CD)}
    Po stwierdzeniu wystąpienia kolizji wysyłany jest sygnał
    zagłuszający, stacja czeka losową liczbę szczelin czasowych
    i ponawia próbę transmisji. Jeśli po 15 próbach się nie powiedzie,
    to zgłasza błąd transmisji.
\end{document}
